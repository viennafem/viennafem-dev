
\chapter{Finite Element Simulations}  \label{chap:fem}

In this chapter all steps required for solving a PDE using the finite element method in {\ViennaFEM} are discussed.
Basically, the discussion follows the tutorials in \lstinline|examples/tutorial/|. 
Familiarity with the very basics of the finite element method is expected. However, since {\ViennaFEM} aims at hiding unimportant low-level details from the
user, no detailed knowledge is required.

\section{Grid Setup}
The first step is to set up the discrete grid using {\ViennaGrid},
which is accomplished by reading the mesh from a file. 
File readers for the VTK format \cite{VTK,VTKfileformat} and the Netgen legacy format \cite{netgen} are provided by {\ViennaGrid}.
The domain type is instantiated by first retrieving the domain type, and then by instantiating the domain as usual:
\begin{lstlisting}
 //retrieve domain type:
 typedef viennagrid::config::triangular_2d                 ConfigType;
 typedef viennagrid::result_of::domain<ConfigType>::type   DomainType;

 DomainType my_domain;
 viennagrid::io::netgen_reader my_reader;
 my_reader(my_domain, "../examples/data/square512.mesh");
\end{lstlisting}
Here, the Netgen file reader is used to read the mesh '\texttt{square512.mesh}' included in the {\ViennaFEM} release.

\TIP{ The easiest way of generating meshes for use within {\ViennaFEM} is to use Netgen \cite{netgen}, which also offers a graphical user interface. }


\section{Boundary Conditions}
The next step is to set boundary conditions for the respective partial differential equation.
Only Dirichlet boundary conditions and homogeneous Neumann are supported in {\ViennaFEMversion}.
Inhomogeneous Neumann boundary conditions as well as Robin-type boundary conditions are scheduled for future releases.

Boundary conditions are imposed by interpolation at boundary vertices. Using {\ViennaGrid}, this is achieved by iterating over all vertices and
calling the function \lstinline|set_dirichlet_boundary()| from {\ViennaFEM} for the respective vertices. For example, to specify inhomogeneous Dirichlet
boundary conditions $ = 1$ for all vertices with $x$-coordinate equal to zero, the required code is:
\begin{lstlisting}
 typedef viennagrid::result_of::ncell_range<DomainType, 0>::type  
                                                           VertexContainer;
 typedef viennagrid::result_of::iterator<VertexContainer>::type 
                                                           VertexIterator;

 VertexContainer vertices = viennagrid::ncells<0>(my_domain);
 for (VertexIterator vit = vertices.begin();
      vit != vertices.end();
      ++vit)
 {
   if ( (*vit)[0] == 0.0 )
    viennafem::set_dirichlet_boundary(*vit, 1.0, 0);
 }
\end{lstlisting}
The first argument to \lstinline|set_dirichlet_boundary| is the vertex, the second argument is the Dirichlet boundary value, and the third argument is
the simulation ID. The simulation ID allows to distinguish between different finite element solutions of possibly different partial differential equations and
should be chosen uniquely for each solution to be computed. However, it is not required to pass a simulation ID to \lstinline|set_dirichlet_boundary|, in which
case it defaults to zero.

The identification of boundary vertices may not be possible in the cases where the simulation domain is complicated. In such cases, the indices of boundary
vertices may be either read from a separate file, or boundary nodes are flagged in the VTK file already. In addition, boundary and/or interface detection
functionality from {\ViennaGrid} can be used.


\section{PDE Specification}
The actual PDE to be solved is specified using the symbolic math library {\ViennaMath}. 
Either the strong form or the weak form can be specified. For the case of the Poisson equation
\begin{align}
 \Delta u = -1 \ ,
\end{align}
the representation using {\ViennaMath} is
\begin{lstlisting}
 function_symbol u;
 equation poisson_eq = make_equation( laplace(u), -1);
\end{lstlisting}
where all functions and types reside in the namespace \lstinline|viennamath|.

It is also possible to supply an inhomogeneous right hand side $f$ instead of a constant value $-1$.
In {\ViennaMathversion}, $f$ is required to be constant within each cell of the mesh.
The values of $f$ are accessed via objects of type \lstinline|cell_quan<T>|, where \lstinline|T| is the cell type of the {\ViennaGrid}
domain:
\begin{lstlisting}
  viennafem::cell_quan<CellType>  f;
  f.wrap_constant( data_key );  
\end{lstlisting}
Here, \lstinline|data_key| denotes the key to be used in order to access the values of $f$ of type \lstinline|double| from the respective cell.

\TIP{Have a look at \lstinline|examples/tutorial/poisson_cellquan_2d| for an example of using \lstinline|cell_quan|.}

\section{Launching a Simulation}




\section{Postprocessing Results}



\section{{\LaTeX} Simulation Protocol}



